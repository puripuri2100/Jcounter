\documentclass{article}
\usepackage{luatexja}
\usepackage{jcounter}
\begin{document}
\section{漢数字}
\verb|\jkansuji{算用数字}|で行うことができる。最大値は\LaTeX{}が扱える最大値の関係で\jkansuji{2147483647}となっている。実用上支障が出ることは少ないと思われる。

\begin{table}[h]
\centering
    \begin{tabular}{ll}
        入力&出力\\\hline
        \verb|\jkansuji{1234}|&\jkansuji{1234}\\
        \verb|\jkansuji{2147483647}|&\jkansuji{2147483647}
    \end{tabular}
\end{table}

\section{全角数字}
\verb|\jzenkakusuji{算用数字}|で行うことができる。最大値の制限はないはず。

\begin{table}[h]
\centering
    \begin{tabular}{ll}
        入力&出力\\\hline
        \verb|\jzenkakusuji{1234}|&\jzenkakusuji{1234}\\
        \verb|\jzenkakusuji{2147483647}|&\jzenkakusuji{2147483647}\\
        \verb|\jzenkakusuji{9999999999}|&\jzenkakusuji{9999999999}
    \end{tabular}
\end{table}

\section{いろは}
いろは変換には2通りある。カタカナとひらがなである。

\subsection{カタカナ}
\verb|\jiroha{算用数字}|で行うことができる。最大値は48。「ン」までありますが、異論は認めません。

\begin{table}[h]
\centering
    \begin{tabular}{ll}
        入力&出力\\\hline
        \verb|\jiroha{1}|&\jiroha{1}\\
        \verb|\jiroha{48}|&\jiroha{48}
    \end{tabular}
\end{table}

\subsection{ひらがな}
\verb|\jiroha{算用数字}|で行うことができる。最大値は48。「ん」までありますが、異論は認めません。

\begin{table}[htb]
\centering
    \begin{tabular}{ll}
        入力&出力\\\hline
        \verb|\jIROHA{1}|&\jIROHA{1}\\
        \verb|\jIROHA{48}|&\jIROHA{48}
    \end{tabular}
\end{table}
\end{document}